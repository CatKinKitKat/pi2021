\documentclass[a4paper,10pt]{article}
\usepackage[portuguese]{babel}
\usepackage[utf8x]{inputenc}
\usepackage[T1]{fontenc}
\usepackage{newtxtext,newtxmath} %TimesNewRoman
\usepackage[sectionbib, numbers, comma, sort]{natbib}
\usepackage{chapterbib}
\usepackage[a4paper,top=3cm,bottom=2cm,left=3cm,right=3cm,marginparwidth=2cm]{geometry}
\usepackage{amsmath}
\usepackage{natbib}
\usepackage{graphicx}
\usepackage[svgnames]{xcolor}
\usepackage[colorinlistoftodos]{todonotes}
\usepackage[colorlinks=true, allcolors=ipbbrown]{hyperref}
% \usepackage{minted}
% \usepackage{adjustbox}

% school colours
\definecolor{ipbgreen}{RGB}{166,204,59}
\definecolor{ipbbrown}{RGB}{153,80,42}

% code block style
% \usemintedstyle{arduino}

\title{\includegraphics[scale=0.5]{ipbeja_logo.png}\\[0.5cm]Sistema de apoio à promoção do turismo rural\\Fase de ETL} % Doc name
\author{Gonçalo Amaro -- 17440,\\ Pedro Tomás -- 18962,\\ Vítor Abreu -- 18966} % Doc's author/s
\date{15 de Dezembro, 2021} % Doc date

\def\blankpage{%
      \clearpage%
      \thispagestyle{empty}%
      \addtocounter{page}{-1}%
      \null%
      \clearpage}

\begin{document}
\bibliographystyle{IEEEtranN}

\maketitle

\blankpage{}

{
  \hypersetup{linkcolor=black}
  \tableofcontents
}

\newpage

\section{Introdução}

O processo de "ETL" tal como o nome indica trata-se de uma metodologia de tradamento de dados, isto é, processos de extração, transformação e carregamento dos dados armazenados.
\begin{itemize}
  \item  O processo de extração trata-se da recolha de dados de múltiplas "origens".
  \item O processo de transformação converte os dados para os tornar adequados ao modelo de dados definido de acordo com o objetivos do negócio. Tendo como objetivo alcançar resultados "homogéneos".
  \item Finalmente, a fase de carregamento, os dados são carregados para um armazém de dados "destino" com o objetivo de visualiozar os dados
\end{itemize}

\section{Planeamento ETL para o nosso trabalho}

Uma vez que já foi explicado o que trata o ETL e tendo em conta o que foi realizado no nosso projeto, a primeira fase de extração de dados já se encontra finalizada, e a parte de transformação está em andamento, já que todos os resultados armazenados ainda não foram tratados, existindo ficheiros .csv com vários caracteres errados ou estranhos pelo leitor de ficheiros. Após finalizada a tranformação, iremos levar os resultados para a parte de carregamento e tratar os dados em gráficos para poder visualizar e chegar a conclusões certas.


\newpage



\section{Webgrafia}

\renewcommand{\bibsection}{}
\bibliography{etl.bib}

\end{document}
