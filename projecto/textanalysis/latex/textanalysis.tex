\documentclass[a4paper,10pt]{article}
\usepackage[portuguese]{babel}
\usepackage[utf8x]{inputenc}
\usepackage[T1]{fontenc}
\usepackage{newtxtext,newtxmath} %TimesNewRoman
\usepackage[sectionbib, numbers, comma, sort]{natbib}
\usepackage{chapterbib}
\usepackage[a4paper,top=3cm,bottom=2cm,left=3cm,right=3cm,marginparwidth=2cm]{geometry}
\usepackage{amsmath}
\usepackage{natbib}
\usepackage{graphicx}
\usepackage[svgnames]{xcolor}
\usepackage[colorinlistoftodos]{todonotes}
\usepackage[colorlinks=true, allcolors=ipbbrown]{hyperref}
\usepackage{minted}
\usepackage{adjustbox}

% school colours
\definecolor{ipbgreen}{RGB}{166,204,59}
\definecolor{ipbbrown}{RGB}{153,80,42}

% code block style
\usemintedstyle{arduino}

\title{\includegraphics[scale=0.5]{ipbeja_logo.png}\\[0.5cm]Sistema de apoio à promoção do turismo rural\\Fase de Análise de Texto via Machine Learning} % Doc name
\author{Gonçalo Amaro -- 17440,\\ Pedro Tomás -- 18962,\\ Vítor Abreu -- 18966} % Doc's author/s
\date{XX de Janeiro, 2022} % Doc date

\def\blankpage{%
      \clearpage%
      \thispagestyle{empty}%
      \addtocounter{page}{-1}%
      \null%
      \clearpage}

\begin{document}
\bibliographystyle{IEEEtranN}

\maketitle

\blankpage{}

{
  \hypersetup{linkcolor=black}
  \tableofcontents
}

\newpage

\section{Introdução}

Após descrito os fundamentos de ETL e uma vez que a fase de estamos atualmente é a de transformação será necessário ser usado um tipo de processos bastante conhecido chamado de ``extração de palavras chave'' e ``análise de sentimentos''.
Estes processos consistem:

\begin{itemize}
  \item \text{retirar do texto as palavras mais importantes que sejam capazes de descrever o estabelecimento.}
  \item \text{caracterizar o sentimento do texto, ou seja, saber se o texto é positivo ou negativo.}
\end{itemize}

\section{Análise de Texto}

Uma Análise de Texto é um processo de extração de palavras chave e análise de sentimento. Basicamente os processos acima descritos.
Mais especificamente a extração de palavras chave é feita através de um algoritmo de Machine Learning que é chamado de ``bag of words'' ou ``bag of features''; e a análise de sentimento é feita através de um algoritmo de Machine Learning que é chamado de ``Naive Bayes'' ou ``Naive Bayes Classifier'', ou usando ``transformers'' ou ``transformer'' que é um algoritmo de Machine Learning que é chamado de ``Tf-Idf'' ou ``Term Frequency - Inverse Document Frequency''.

O ultimo caso foi descrito com duas opções porque como poderão verificar mais à frente, a nossa tentativa de ``Naive Bayes'' não foi bem sucedida, e por isso foi usado um ``transformer'' com um dataset chamado de ``Bert'' ou ``Bert For Classification'', este foi criado pela ``Google''.

Para a realização dos processos acima descritos usaremos a linguagem de programação ``Python'' e variadas bibliotecas.
Das quais:

\begin{itemize}
  \item \textbf{nltk} -- \href{https://pypi.org/project/nltk/}{Natural Language Toolkit}.
  \item \textbf{sklearn} -- \href{https://pypi.org/project/scikit-learn/}{Scikit-learn}.
  \item \textbf{numpy} -- \href{https://pypi.org/project/numpy/}{Numpy}.
  \item \textbf{pandas} -- \href{https://pypi.org/project/pandas/}{Pandas}.
  \item \textbf{transformers} -- \href{https://pypi.org/project/transformers/}{Transformers}.
  \item \textbf{torch} -- \href{https://pypi.org/project/torch/}{PyTorch}.
  \item \textbf{yake} -- \href{https://pypi.org/project/yake/}{YAKE!}.
\end{itemize}

Será feita a realização de um script por cada processo, que será transformado num notebook, via \textit{p2j}, para que possa ser documentado.

\subsection{Extração de Palavras Chave}

Aqui foi executado o processo de extração de palavras chave, que consistiu em usar o YAKE!, que é uma biblioteca de autores portugueses e um japonês que foi criada para extrair palavras chave de um texto. Os seus autores são residentes em: Instituto Politécnico de Tomar, Universidade da Beira Interior, Universidade do Porto, INESC TEC e Universidade de Kyoto.

A razão especifica do uso desta biblioteca ao invés de fazer de raiz, foi o facto acima descrito, em nome de apoio ao uso de trabalhos de autoria portuguesa.

\subsubsection{Processo de Execução}

Este ``script'' funciona da seguinte forma:

\begin{itemize}
  \item Indicamos o caminho da pasta onde se encontra o texto a ser analisado.
  \item Importa todos os \text{.csv} de reviews que se encontram na pasta.
  \item Por cada \text{.csv} de reviews, extrai as palavras chave.
  \item Exporta as palavras chave para um \text{.csv}.
\end{itemize}

\subsubsection{Processo de Desenvolvimento}

\subsection{Análise de Sentimento}

\subsubsection{Processo de Execução}

Este ``script'' funciona da seguinte forma:

\begin{itemize}
  \item Indicamos o caminho da pasta onde se encontra o texto a ser analisado.
  \item Importa todos os \text{.csv} de reviews que se encontram na pasta.
  \item Por cada \text{.csv} de reviews, faz a análise de sentimento.
  \item Exporta os sentimentos para um \text{.csv}.
\end{itemize}

\subsubsection{Processo de Desenvolvimento}

\section{Resultados}

\subsection{Extração de Palavras Chave}

\subsection{Análise de Sentimento}

\section{Conclusão}

\section{Webgrafia}

\renewcommand{\bibsection}{}
\bibliography{textanalysis.bib}

\end{document}
