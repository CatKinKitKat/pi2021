\documentclass[a4paper,10pt]{article}
\usepackage[portuguese]{babel}
\usepackage[utf8x]{inputenc}
\usepackage[T1]{fontenc}
\usepackage{newtxtext,newtxmath} %TimesNewRoman
\usepackage[sectionbib, numbers, comma, sort]{natbib}
\usepackage{chapterbib}
\usepackage[a4paper,top=3cm,bottom=2cm,left=3cm,right=3cm,marginparwidth=2cm]{geometry}
\usepackage{amsmath}
\usepackage{natbib}
\usepackage{graphicx}
\usepackage[svgnames]{xcolor}
\usepackage[colorinlistoftodos]{todonotes}
\usepackage[colorlinks=true, allcolors=ipbbrown]{hyperref}
\usepackage{minted}
\usepackage{adjustbox}

% school colours
\definecolor{ipbgreen}{RGB}{166,204,59}
\definecolor{ipbbrown}{RGB}{153,80,42}

% code block style
\usemintedstyle{arduino}

\title{\includegraphics[scale=0.5]{ipbeja_logo.png}\\[0.5cm]Sistema de apoio à promoção do turismo rural\\Fase de Análise de Texto via Machine Learning} % Doc name
\author{Gonçalo Amaro -- 17440,\\ Pedro Tomás -- 18962,\\ Vítor Abreu -- 18966} % Doc's author/s
\date{XX de Janeiro, 2022} % Doc date

\def\blankpage{%
      \clearpage%
      \thispagestyle{empty}%
      \addtocounter{page}{-1}%
      \null%
      \clearpage}

\begin{document}
\bibliographystyle{IEEEtranN}

\maketitle

\blankpage{}

{
  \hypersetup{linkcolor=black}
  \tableofcontents
}

\newpage

\section{Introdução}

Após descrito os fundamentos de ETL e uma vez que a fase de estamos atualmente é a de transformação será necessário ser usado um tipo de processos bastante conhecido chamado de "keyword extraction" este processo consiste essencialmente em retirar do texto as palavras mais importantes que sejam capazes de o descrever, ficando assim com o essencial e de uma maneira mais clara e simplificada.

\section{Planeamento}

Para a realização dos processos de "keyword extraction" usaremos a linguagem de programação "Python" e uma possível opção será a realização de um script capaz de remover e limpar todo o texto armazenado da fase de extração, deixando-o assim apto para a fase de carregamento.

\section{Webgrafia}

\renewcommand{\bibsection}{}
\bibliography{textanalysis.bib}

\end{document}
