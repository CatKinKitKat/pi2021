\documentclass[a4paper,10pt]{article}
\usepackage[portuguese]{babel}
\usepackage[utf8x]{inputenc}
\usepackage[T1]{fontenc}
\usepackage{newtxtext,newtxmath} %TimesNewRoman
\usepackage[sectionbib, numbers, comma, sort]{natbib}
\usepackage{chapterbib}
\usepackage[a4paper,top=3cm,bottom=2cm,left=3cm,right=3cm,marginparwidth=2cm]{geometry}
\usepackage{amsmath}
\usepackage{natbib}
\usepackage{graphicx}
\usepackage[svgnames]{xcolor}
\usepackage[colorinlistoftodos]{todonotes}
\usepackage[colorlinks=true, allcolors=ipbbrown]{hyperref}
\usepackage{minted}
\usepackage{adjustbox}

% school colours
\definecolor{ipbgreen}{RGB}{166,204,59}
\definecolor{ipbbrown}{RGB}{153,80,42}

% code block style
\usemintedstyle{arduino}

\title{\includegraphics[scale=0.5]{ipbeja_logo.png}\\[0.5cm]Sistema de apoio à promoção do turismo rural\\Fase de Análise de Texto via Machine Learning} % Doc name
\author{Gonçalo Amaro -- 17440,\\ Pedro Tomás -- 18962,\\ Vítor Abreu -- 18966} % Doc's author/s
\date{XX de Janeiro, 2022} % Doc date

\def\blankpage{%
      \clearpage%
      \thispagestyle{empty}%
      \addtocounter{page}{-1}%
      \null%
      \clearpage}

\begin{document}
\bibliographystyle{IEEEtranN}

\maketitle

\blankpage{}

{
  \hypersetup{linkcolor=black}
  \tableofcontents
}

\newpage

\section{Introdução}

\section{Planeamento}

\subsection{Divisão de Tarefas}

\subsection{Tecnologias Usadas}

\subsubsection{Ambientes Virtuais de Python}

\subsubsection{Bibliotecas de Python usadas}

\subsection{Estratégia}

\subsection{Desenvolvimento}

\subsection{Resultados}

\section{Próximos Passos}

\section{Conclusão}

\section{Webgrafia}

\renewcommand{\bibsection}{}
\bibliography{textanalysis.bib}

\end{document}
