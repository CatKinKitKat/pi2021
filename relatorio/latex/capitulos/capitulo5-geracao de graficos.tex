\chapter{Geração de gráficos}
\label{cap5}

Para proceder à análise dos dados, é necessário que os dados sejam organizados num formato que permita a leitura e a visualização dos dados facilitada a humanos. Foram gerados os gráficos para a análise de sentimentos e \textit{keywords} em \textit{sets} de totais e de forma temporal.

Para os gráficos de totais, foi utilizado o pacote \textit{matplotlib} \cite{va1} e o \textit{WordCloud} \cite{gfg3} em que foram usados todos os dados disponíveis (das três plataformas). Já os gráficos de análises temporais foram gerados apenas com os dados do \textit{TripAdvisor}, visto que não só eram os mais completos e extensos de todas as categorias de estabelecimento, como também foi possível extrair as datas de criação dos \textit{reviews}.

Com gráficos de totais, queremos dizer que nos dados apresentados e nas análises não é considerado o desenvolvimento temporal, mas sim o desenvolvimento de um total de \textit{reviews}, ou seja, o mês e ano são descartados.

\section{Metodologia}

Para a geração dos gráficos de totais, foi utilizado o pacote \textit{matplotlib} \cite{va1} e o \textit{wordcloud} \cite{gfg3} num \textit{script Python}, que iterava sobre os dados disponíveis e gerava os gráficos para cada estabelecimento.

Os tipos de gráficos gerados são: gráficos circulares de sentimentos, gráficos de \textit{keywords} e nuvens de \textit{keywords}. Os quais demonstram a quantidade de sentimentos positivos e negativos, as dez \textit{keywords} mais frequentes e as nuvens de \textit{keywords} limitadas até cem palavras.

\subsection{Execução}

Para cada tipo de estabelecimento, de cada plataforma, foi accionada as rotinas de geração dos gráficos. As quais foram exportadas em formato de imagem. Iterando sobre os tipos de estabelecimento e as plataformas, em que itera sobre os dados disponíveis, segmentados em sentimentos e \textit{keywords}, que passam por uma e duas funções respectivamente. Gerando e exportando \textit{.jpg} dos gráficos e nuvens.

\subsection{Resultados}

Como se podem verificar nestas três imagens seguintes (\hyperref[fig:exemplofig21sentimentsTrip]{\textbf{5.1}}, \hyperref[fig:exemplofig21keywordsTrip]{\textbf{5.2}} e \hyperref[fig:exemplofig21cloudTrip]{\textbf{5.3}}), os gráficos de totais apresentam um desenvolvimento de sentimentos e \textit{keywords}, que demonstra um sentimento total positivo no nosso turismo, e \textit{keywords} que apresentam a satisfação com o serviço ou uma característica do estabelecimento.

\begin{figure}[!htb]
\centering
\includegraphics[width=14cm]{figuras/TripAdvisor/Hotels/hotel21_sentiments.jpeg}
\caption{Gráfico circular gerado baseando-se nos sentimentos dados da plataforma \textit{TripAdvisor} referente à Herdade das Barradas da Serra}
\label{fig:exemplofig21sentimentsTrip}
\end{figure}

\begin{figure}[!htb]
\centering
\includegraphics[width=14cm]{figuras/TripAdvisor/Hotels/hotel21_keywords.jpeg}
\caption{Gráfico circular gerado baseando-se nas \textit{keywords} mais usadas da plataforma \textit{TripAdvisor} referente à Herdade das Barradas da Serra}
\label{fig:exemplofig21keywordsTrip}
\end{figure}

\begin{figure}[!htb]
\centering
\includegraphics[width=14cm]{figuras/TripAdvisor/Hotels/hotel21_keywordcloud.jpeg}
\caption{Gráfico de \textit{keywords} e nuvens de \textit{keywords} contendo as \textit{keywords} mais usadas da plataforma \textit{TripAdvisor} referente à Herdade das Barradas da Serra}
\label{fig:exemplofig21cloudTrip}
\end{figure}

\clearpage
\section{Reorganização dos dados (\textit{TripAdvisor} apenas)}

Para as análises temporais e para o transporte de dados, foi necessário criar uma base de dados organizada e relacional, que garantisse a integridade dos dados e a sua coerência na importação dos mesmos para o \textit{PowerBI}. A importação de apenas ficheiros \textit{.csv} não nos apresenta relações entre tabelas, nem como quais possíveis relações funcionariam.

Para isso foram usados os pacotes \textit{pandas} para o \textit{import} dos \textit{.csv} em \textit{DataFrames} e \textit{sqlite3} para a criação da base de dados \textit{SQLite3}.

\subsection{Metodologia}

Inicialmente, foi criada uma base de dados \textit{SQLite3}, que será usada para armazenar os dados. A base de dados \textit{SQLite3} foi criada e as tabelas foram criadas, com os campos correspondentes a cada coluna do ficheiro \textit{.csv}, dos quais os dados foram importados via \textit{pandas}. 

Os \textit{DataFrames} de \textit{pandas} oferecem uma maneira de aceder e manipular dados, e também fornecem uma maneira de criar e manipular tabelas. Todas as operações de manipulação de dados são feitas através de funções de \textit{DataFrame}.

Seguidamente, é empurrado os dados dos \textit{DataFrames} correspondentes às tabelas para a base de dados. Com esta base de dados, é possível fazer consultas e manipulações de dados, tal como também exportar os dados organizados para ficheiros \textit{.csv} de forma a que possam ser usados em outros programas, tais como o \textit{R}, o \textit{Excel}, o \textit{PowerBI}, etc.

\subsection{Execução e Resultados}

Foi feito um \textit{script} para executar a criação da base de dados \textit{SQLite3}, e para o \textit{import} dos dados dos \textit{DataFrames} para a base de dados. O qual foi executado com sucesso, como podemos ver na base de dados e nos ficheiros \textit{.csv}.

\section{Gráficos temporais (\textit{TripAdvisor} apenas)}

Com gráficos temporais, queremos dizer que nos dados apresentados e nas análises é considerado o desenvolvimento temporal, quer isto dizer que as \textit{reviews} mostradas ao longo do tempo tornam possível verificar as datas em que foram escritas e a quantidade que cada hotel/restaurante/atracção recebeu ao longo dos anos e meses.

\subsection{Metodologia}

Para a geração dos gráficos temporais, foi utilizado o \textit{software PowerBI}, que utilizava os ficheiros .csv gerados e organizados previamente vindos da base de dados, contendo todas as informações em ficheiros únicos.
Os gráficos gerados são: gráficos circulares e de tabelas. Os quais demonstram a quantidade de sentimentos e \textit{keywords} usadas ao longo do tempo por cada hotel, divididos por anos e meses e também por cada hotel.

\subsection{Execução}

Os ficheiros \textit{.csv} que contêm as informações relativas a todas as \textit{keywords} e sentimentos, foram importados para o \textit{software PowerBI} e posteriormente organizados da maneira que o grupo achou mais conveniente para que os gráficos ficassem o mais apresentáveis e visivelmente mais fáceis para analisar os dados. Posteriormente os gráficos circulares e de tabelas foram exportados para \textit{.jpg} e guardados. 

\subsection{Resultados}

Como se podem verificar nestas três seguintes imagens (\hyperref[fig:exemplofigqntsntmyear]{\textbf{5.4}}, \hyperref[fig:exemplofigqntyearbus]{\textbf{5.5}} e \hyperref[fig:exemplofigposneg]{\textbf{5.6}}), os gráficos temporais apresentam um desenvolvimento de sentimentos e \textit{keywords} ao longo do tempo bastante positivo revelando-se um óptimo ponto para o nosso turismo e \textit{keywords} que mostram bastante agrado. 

\begin{figure}[!htb]
\centering
\includegraphics[width=14cm]{figuras/NrReviewsPerYear/TableGraph6.PNG}
\caption{Gráfico de tabelas gerado baseando-se em todos os sentimentos dados da plataforma \textit{TripAdvisor} ao longo dos anos}
\label{fig:exemplofigqntsntmyear}
\end{figure}

\begin{figure}[!htb]
\centering
\includegraphics[width=14cm]{figuras/NrReviewsPerYear&BusinessType/8.PNG}
\caption{Gráfico de tabelas gerado baseando-se em todos os sentimentos da plataforma \textit{TripAdvisor} referente a cada hotel com o decorrer dos anos}
\label{fig:exemplofigqntyearbus}
\end{figure}

\begin{figure}[!htb]
\centering
\includegraphics[width=14cm]{figuras/Pos&NegSentiments/TableGraph4.PNG}
\caption{Gráfico de tabelas gerado baseando-se na quantidade de sentimentos da plataforma \textit{TripAdvisor} positivos e negativos ao longo do tempo para cada hotel}
\label{fig:exemplofigposneg}
\end{figure}