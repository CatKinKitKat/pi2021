\chapter{Extracção de Palavras-Chave}
\label{cap5}

\section{Extracção de Palavras-Chave}

A extracção de palavras-chave é uma técnica de extracção de informações que consiste em extrair palavras-chave de um texto.

Geralmente, as palavras-chave são utilizadas para identificar o conteúdo de um documento, ou seja, para identificar o que o documento contém. No nosso caso, as palavras-chave são utilizadas para identificar pontos fulcrais da recepção dum cliente turístico num estabelecimento turístico de Beja (hotel, atracção, restaurante, etc).

Com estas palavras-chave, é possível identificar o que o cliente quer, ou seja, o que ele quer ver, o que ele quer comer, o que ele quer fazer, ou o nível de satisfação com o serviço prestado.

\subsection{Metodologia}

Para a extracção de palavras-chave, utilizamos uma biblioteca de \textit{Python} chamada \textit{YAKE!}. Esta é um pipeline de processamento de linguagem natural, que utiliza um algoritmo personalizado descendente do \textit{Naïve Bayes} para extrair palavras-chave de um texto.

Para a extracção de palavras chaves de cada \textit{review}, utilizamos o \textit{YAKE!} iterativamente por cada \textit{review}, alimentando-o com o seu conteúdo textual, após o pre-processamento textual de todas as \textit{reviews} (a qual especificamente foi usada a versão sem diacríticos).

O output da alimentação do \textit{YAKE} é um par de palavras-chave e seus respectivos pesos, que são armazenados num dicionário. O dicionário é ordenado pelo valor de seus pesos, e o número de palavras-chave extraídas é limitado ao numero de \textit{reviews} que foram utilizadas para a extracção por cada estabelecimento.

\subsubection{\textit{YAKE!}}

Esta biblioteca é um software livre, e pode ser obtida via \textit{pip} ou via GitHub, que é uma ferramenta de extracção de palavras-chave desenvolvida por autores portugueses (e um japonês) da Universidade do Porto, Politécnico de Tomar, e da Universidade da Beira Interior (e da Universidade de Kyoto).

O sua utilização pode ser simples, basta instalar a biblioteca e executar o seguinte comando: \texttt{yake.KeywordExtractor(lan="pt").extract\textunderscore keywords(text)}

No entanto esta pode ser mais complexa caso seja necessário optimizar a extracção de palavras-chave, com os seus variados parâmetros opcionais.

Este funciona da seguinte forma: quando recebe um texto, vai testar todas as palavras do texto, com uma determinada formula, e guardar o peso da palavra, e a palavra em si, no final expele um dicionário com as palavras-chave e seus respectivos pesos.

A formula falada anteriormente é:
\begin{math}
    S(kw) = \frac{\prod_{w \in kw}^{} S(kw)}{TF(kw) * \sum_{w \in kw}^{}S(w)}
\end{math}

Mais especificamente este modulo é uma forma mais delicada e avançada de um classificador \textit{Naïve Bayes} o qual será mais e melhor explicado no capitulo seguinte onde procedemos ao desenvolvimento de um para efeitos de analise de sentimentos.

\subsubsection{Execução}

Foi criado um \textit{notebook} de \textit{Jupyter} o qual contém o código usado para a extracção de palavras via \textit{YAKE!}. Cada bloco de código está sobreposto por um bloco de \textit{markdown}, que é um comentário. Os comentários são usados para explicar o que cada bloco de código faz.

As rotinas de extracção de palavras-chave são: iterativamente, para cada ficheiro \textit{.csv} dentro da pasta indicada, importar via \textit{DataFrame} de \textit{pandas}. O qual \textit{DataFrame} é um conjunto de dados, como uma tabela de dados, e contém uma coluna com os \textit{reviews}. O \textit{DataFrame} é iterado na sua coluna única, e o seu conteúdo é passado para o \textit{YAKE!}. O qual exporta o resultado para um ficheiro \textit{.csv}, que é um ficheiro \textit{.csv} com uma coluna com as palavras-chave e outra com o seu peso.

O \textit{notebook} e o código estão disponíveis nos anexos.

\subsection{Resultados}

Os resultados obtidos detém um sentimento misto no grupo. Estas palavras chave por vezes não são palavras únicas, mas são expressões que são frequentes. Muitas palavras únicas aparecem lado a lado das expressões, o que dá uma noção de repetição ou confirmação de resultado.
