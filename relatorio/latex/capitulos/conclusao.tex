 \chapter{Conclusão}
\label{cap10}

Uma análise de dados apresenta sempre uma oportunidade de crescimento, em especial num meio de turismo rural (ou semi-rural) em que os habituais clientes podem (e vão) usar redes sociais e plataformas de comentário de prestação de serviço, para agregar dados comportamentais e percepção pública das ofertas turísticas.

Neste trabalho foi apresentado um agregado exemplar desse tipo de dados e análise o qual foi feito na melhor das formas, dado o conhecimento a nós atribuído pela escola e o tempo disponível. O qual apresentou pouco de informação nova, mas colabora com a percepção pública que se sente nos meios de comunicação verbal subjectiva em locais públicos, ou seja, existe a tendência de haver um maior número de clientes durante as típicas alturas de férias, referimos o ano novo, a Páscoa e em especial o verão.

O que foi dito acima foram as tendências observadas na análise dos dados obtidos nos gráficos gerados para análise temporal, com os dados da plataforma mais completa deste estudo, o \textit{TripAdvisor}. O único caso mais em especial, foi a explosão de \textit{reviews} em Junho e Julho de 2020, que coincide com a abertura da época balnear após desconfinamento (COVID-19) no dia 6 de Junho de 2020, que embora fazendo \textit{cross-referencing} com os dados das \textit{keywords} da mesma análise temporal, não nos dê informação relevante, a coincidência é meramente atractiva e oferece uma fácil explicação.

Sendo assim podemos concluir que foi um trabalho que corrobora as noções e conhecimento público da área, tal como os dados das análises anuais oferecidas pelo \textit{Booking}. Graças a isto obtemos uma noção da metodologia desta área profissional e os nossos avaliados terão uma noção da qualidade de trabalho produzida por nós.

Sumarizamos assim que nesta secção de turismo rural (e semi-rural), da região de Beja, tem por norma uma boa prestação de serviço, ou pelo menos essa noção é transposta na mente popular de quem visita, e que não existem anomalias na distribuição de visitantes nem de tendências sentimentais ao longo do ano nem ao longo das décadas, mantendo-se previsível mas agradável.

\textbf{UC úteis para a elaboração deste Projecto:} \textit{PE, BD1, LP, ES, TWAM, SI}
